%\documentclass[10pt, conference, compsocconf]{IEEEtran}
\documentclass[10pt, conference, compsocconf]{IEEEtran}
\usepackage{color}
\usepackage{supertabular}
\usepackage{graphicx}
\usepackage{flushend}
\usepackage{cite}
\usepackage{subfigure}
\usepackage{comment}
\usepackage[cmex10]{amsmath}


\bibliographystyle{IEEEtran}

\begin{document}

\newcommand {\xtq}[1] {\begin{color}{red}{[XTQ : #1]}\end{color}}
\newcommand{\kento}[1]{{\color{blue} [Kento: {\it #1}]}}
\newcommand {\xtqmod}[1] {\begin{color}{magenta}{[XTQ MODIFICATION: #1]}\end{color}}

%\newcommand {\xtq}[1] {}
%start title
%\title{Studies of Cloud Based Burst Buffers to Burst Data Intensive Application on
%the Cloud}
\title{Explorations of Cloud-based Burst Buffers for I/O Acceleration}%Data Caching}%Data Intensive
% Applications }

\author{
\IEEEauthorblockN{Tianqi Xu}
	\IEEEauthorblockA{Dept. of Mathematical\\and Computing Sciences\\
		Tokyo Institute of Technology\\
		2-12-1-W8-33, Ohokayama,\\
		Meguro-ku, Tokyo 152-8552 Japan\\
		\kento{Use titech email address ! } \\
	Email: Jyo.t.aa@m.titech.ac.jp}
	\and
	\IEEEauthorblockN{Kento Sato}
	\IEEEauthorblockA{Center for Applied Scientific Computing\\
		Lawrence Livermore National Laboratory\\
		Livermore, CA 94551 USA\\
	Email: kento@llnl.gov}
	\and
	\IEEEauthorblockN{Satoshi Matsuoka}
	\IEEEauthorblockA{Global Scientific \\Information and Computing Center\\
		Tokyo Institute of Technology\\
		2-12-1-W8-33, Ohokayama,\\
		Meguro-ku, Tokyo 152-8552 Japan\\
	Email: matsu@is.titech.ac.jp}
}

\maketitle


%start abstract
\begin{abstract}

Cloud computing offers high computational resources, scalability, as well as
ease of access.
%Such cloud environments offer those who have limited local computational
%resources a opportunity to run large scale applications.
Such cloud environments provide the users with virtually unlimited computational
resources to run HPC applications at larger scale than what in-house
systems can provide. 
%Current data intensive large scale applications generate TB
%intermediate data shared among hundreds of compute nodes, so the shared storage
% throughput becomes critical part.
Since large scale data intensive applications typically generate huge amount of
intermediate data, and shared by hundreds and thousands of compute nodes,
data intensive applications require high I/O throughput to shared storage.
%However compare to the throughput in HPC system, the shared storage
%system in cloud environment can only achieve a quite low I/O throughput,
% becoming the performance bottleneck for data intensive applications.
However, current shared storage in cloud environments can not provide
enough I/O throughput for data intensive applications.
The low I/O throughput becomes a performance bottleneck.
%Furthermore, according to normal cloud pay-as-you-go pricing,
%longer execution time means more you should pay.
Furthermore, the prolonged execution time incur more cost to users as the most
of cloud providers employ pay-as-you-go pricing modes.
%In this paper, we propose a cloud based
%burst buffer system as a new tier in cloud storage system to absorb I/O
% request.
To accelerate I/O performance, we propose a cloud-based burst buffer system as a
new tier in cloud storage systems.
%We use several nodes as a burst buffer nodes,
%take advantage of high throughput inside cloud to buffer application I/O data.
The cloud-based burst buffer system uses computing nodes as burst buffer nodes,
and buffers applications' data in the burst buffer nodes. Because throughput
between compute nodes is much higher than shared storage throughput, we can
accelerate I/O performance for data intensive applications.
%We implemented a
%prototype of our proposal architecture, evaluated in Amazon EC2/S3(one of the
% most common public cloud today).
%We shows our system can achieve a significant improvement in I/O throughput for
% shared storage
%in cloud computing through experimental and simulation evaluation.
To explore the effectiveness of cloud-based burst buffers, 
We implement a prototype, and evaluate the system in Amazon
EC2/S3.
\kento{Let us describe about the modeling.}
The \xtqmod{quantitative} experimental and simulation evaluations show that our prototype system
can achieve significant improvement in I/O throughput.
\kento{Let us include ``quantitative'' results of both experimental results and
simulation results.}
\end{abstract}

%start key word
\begin{IEEEkeywords}
	cloud computing, burst buffer
\end{IEEEkeywords}

\IEEEpeerreviewmaketitle

%section I
\section{Introduction}
\label{sec:introduction}
A growing focus on cloud computing for its high scalability as well as high computational
resources, for example, Amazon provided a high performance computing instance allows user to run
complex science, engineering, and business applications on these instances with a high bandwidth low latency
network, and high compute capabilities.
Cloud provide a fast setup, usually cloud instances can set up in a few minutes, this
make the on-demand usage available.
Such cloud environment is suitable for large scale
computation, by using cloud users don't need a large scale supercomputer and only need to pay
what they run, instead of dealing with expensive hardware ,software and machine maintenance.

Large scale applications often require high I/O
throughput and data sharing between different nodes, different work flow and even different jobs.
In such cases, applications require not only high bandwidth network as well as a high throughput
shared storage in order to share data between different work flow or difference jobs.
Applications like Montage Pov Ray generate huge size of intermediate file
to shared between different nodes in different work flow.
Such applications are now running majorly on high performance computing system like supercomputers, with a GB/s level shared storage as well as high interconnection network.
Furthermore, large scale applications require for a large amount of compute node as well as a
running time from several hours to several days, machine failure is Inevitable and checkpoint will be required to enable
fast recovery from machine failure\cite{checkpointing}, in order to make a timely checkpoint, a fast global
shared storage is indispensable.

However, In current cloud environment, the global shared storage can not meet the demand.
For example, in Amazon public cloud system, there are a famous shared storage called \emph{Amazon
Simple Storage Service}(Amazon S3), however compared to the throughput of shared storage system in
high performance computing system, Amazon S3 can only achieve a extremely low throughput, \xtq{figure}as well as has a high latency, since storage machine in Amazon S3 are geographically distributed and connected via Internet.
Also it is difficult to build each compute center a reasonable size data center, since it can not achieve a high utilization of storage system.

In order to solve such problem, we propose a cloud based burst buffer system. Burst buffer system
has been proposed as a new tier in the storage hierarchy to hide the low bandwidth and high latency
of the shared file system by allocate several local nodes as a burst buffer nodes to absorb the I/O
request from applications, our proposal system using such burst buffer system and take advantage of
high throughput, low latency interconnection network inside cloud environment, make some of compute nodes as a I/O burst buffer nodes to buffer I/O data and bursting I/O sensitive applications in cloud environment.
Our proposal system not only target on the work flow applications, all the applications have
I/O data locality or even output a huge size can get speed up on our proposal system.
We implemented a prototype of our proposal system and evaluated on real public cloud environment, as
well as simulated with several applications.
We achieved 8 times speed up on reading as well as 14 times speed up on writing with 8 I/O nodes.

Our contributions can be summarized as following:
\begin{itemize}
	\item A cloud based I/O burst buffer system for increasing data transfer throughput in cloud environment;
	\item A quantitative evaluation of prototype of I/O burst buffer system in Amazon EC2 to show the
	Effectiveness of our proposal system;
    \item A simulation with several applications in Amazon EC2 to show how applications can achieve
    performance burst by using our proposal system;
\end{itemize}
%2
The rest of this paper is organized as follows. 
In Section \ref{sec:background}, we clarify the motivation and the background.
We introduce an overview of the I/O bursting buffer architecture in Section \ref{sec:architecture}, 
and describe the detail about the first prototype implementation of our proposal architecture in order to evaluate its performance in a real environment \ref{sec:implementation}. 
In Section \ref{sec:evaluation}, we present our experimental results based of our performance models. %based on data obtained from several benchmark on TSUBAME V queue and Amazon EC2 
Finally, we show some related work in Section \ref{sec:related work}, and conclusion in Section \ref{sec:conclusion}.


%section II
%to do
\section{background}
\label{sec:background}
\subsection{cloud computing}

\subsection{burst buffer}
In modern high performance systems storage systems can easily become the bottleneck of the whole system\xtq{citation}, 
\subsection{workflow application}
Montage is a toolkit developed by Spitzer Science Center for assembling Flexible Image Transport System (FITS) images into custom mosaics.
It has been built, tested and the output products validated by Montage customers on Unix platforms, including Linux, Solaris, Mac OSX, and IBM AIX.
It has been used to generate mosaics from data released by the Spitzer Space Telescope, the Hubble Space Telescope, the Infrared Astronomical Satellite (IRAS),
the Midcourse Space Experiment (MSX), the Sloan Digital Sky Survey (SDSS), and ground-based telescopes such as the National Optical Astronomy Observatories (NOAO) 4-m telescope and the William Herschel 4-m telescope.

The Persistence of Vision Raytracer, or POV-Ray, is a ray tracing program which generates images from a text-based scene description,
and is available for a variety of computer platforms. It was originally based on DKBTrace, written by David Kirk Buck and Aaron A.
Collins for the Amiga computers.
There are also influences from the earlier Polyray raytracer contributed by its author Alexander Enzmann.
POV-Ray is free and open-source software with the source code available under the AGPLv3.
Many methods for generating the 3-D models are used, including a companion program "moray" for interactive modeling.



%section III
\section{Proposed Cloud-based Buffer Bursts}
\label{sec:architecture}

%\begin
An overview of I/O burst buffer architecture is described in this section.
As we mentioned in the previous section, our model takes advantage of high throughput inside a system, and use buffer queue system in order to increase throughput between two systems.
%Two kinds of buffer are used in our I/O burst buffer architecture, the first one is in client computing node, first buffer user I/O in the same node, another one is in I/O buffer nodes.
The main idea is that some of computing nodes serve as a I/O buffer nodes in each system, for write I/O data, if buffer queue in I/O buffer nodes is not full, data can first be buffered in buffer queue in the same system, and then client can finish write operation without waiting data be finally transferred to storage system.
Actually, since many jobs use multiple nodes work together and the output of some nodes can be the
input of the others, so for most cases, output data will still be buffered in buffer queue until the whole job finished.
As for reading, if that file is stored in the buffer queue, compute nodes can read from I/O nodes through interconnection network.
In other cases (buffer queue is full when issue a write request or requested file is not stored in buffer queue when issue a read request we call it cache miss), a read from or write back operation described below will be executed. 

\subsection{Target Environment}
First, we introduce our target environment.
we target at common cloud environment like Amazon EC2, which compute node and data center are distributed geographically distributed.

\begin{itemize}
	\item All compute nodes are connected by large bandwidth and interconnection network, note network topology maybe different in each system, so topology is not specified here, interconnection network performance is measured by throughput.
	\item There is a shared storage for data sharing inside system, all compute nodes are connected
	with shared storage via Internet or other lower network, also the file system of shared storage is not specified and performance is measured by throughput.
\end{itemize}

\subsection{I/O Burst Buffer Architecture}

\begin{figure}[tb]
	\centering
	\includegraphics[width=8cm]{img/architecture_overview}
	\caption{overall illustrate of I/O Burst Buffer Architecture}
	\label{architecture:overview}
\end{figure}

There are two kinds of nodes in our I/O burst buffer architecture: \emph{compute nodes} and \emph{I/O burst nodes}, compute nodes run user's application and I/O burst nodes.
Public clouds usually provide several type of instance for difference purpose, some are compute
optimized (like C3 instances provided by Amazon) provides a high compute performance, some are memory optimized (such as R3 instances in Amazon EC2) provide a huge amount of memory (up
to 244GB in R3 family), also memory size can be easily expanded by using solid storage driver (SSD)
as a swap device, usually such instances have SSD inside.

Fig.~\ref{architecture:overview} shows an overall illustrate of I/O burst buffer architecture.

%section IV
%working on this section
\begin{figure}
\centering
%\includegraphics{}
\caption{overview of prototype}
\label{overview of prototype}
\end{figure}

In order to evaluate our architecture in a real environment and compare with current architecture, we implemented a prototype. 
In our architecture there are several global data such as total number of burst buffer nodes,
buffered file information. need to be shared between all nodes.
And since in our proposal architecture, numbers of burst buffer nodes can be dynamically determined,
a global scheduler is needed to decide when and how nodes are added and removed, these decision should be made based on global work load and local work load.
Also a scheduler is needed to handle the operation like re-balance of buffered data when nodes addition, and data write back when nodes deletion.

For above reasons in this implementation, we adopted master-worker model, a master is designed to store global information and interactive with client daemon, as well as serve as a scheduler.
First we introduce our target environment:
\begin{itemize}
  \item a common cloud environment, several types of instance are available for using.
  \item all nodes are connected via interconnection network, which is about 1Gb/s or larger, and has a strong scaleab
  \item a shared storage 
\end{itemize}
The architecture of our prototype is described as Figure \ref{overview of prototype}

In our architecture, there are three types of nodes:
\begin{itemize}
	\item A master node manages all IOnodes information, maintains a set of buffered file meta data, handles operation like IOnode addition and deletion etc., and interact with clients.
	\item There are several IOnodes response for actually storing data.
	\item A daemon runs on each client machine serving for communication with
	master and IOnodes, in order to hide architecture detail from user.
\end{itemize}

Details of each nodes will be shown in following subsections:

\subsection{Master Node}
In our master-IOnode model, master node is the supervisor of the entire system,
Master node is in charge of interactive with client, including file opening, reading and writing.

In order to manage the whole system and schedule events like node addition and deletion, master node store following data structure:

\begin{table}
\centering
\begin{tabluar}{c|c}
\hline
I\textbackslash O Node Pool &&
storing registered I\backslash O node information including node id, ip address, available memory, total memory, communication socket\\\hline
buffered file information &&
including file id, size, block size, dirty flag\\\hline
\end{tabluar}
\caption{}
\end{table}

Master functionality:

\begin{table}
\centering
\begin{tabluar}{c|c}
\hline
handle file open operation && if file is buffered, return set of I\textbackslash Onode, else get file size, assigned appropriate number of nodes depends on data layout si
\end{tabluar}
\end{table}
\subsection{I/O Node}

\subsection{Client Daemon}

%section V
\subsection{modeling}
\label{sec:modeling}

In order to simulate the execution of the three applications shown in Table.~\ref{}

%section VI
\section{evaluation}
\label{sec:evaluation}

In this section we introduce the evaluation result of our prototype implementation in Amazon EC2 public cloud.
The cloud environment is shown as Table.\ref{evaluation:amazon_environment}
\begin{table}[h]
\centering
\begin{tabular}{|c|c|}
Region				&		Tokyo		\\
Instance Type		&		m3.xlarge	\\
vCPUs				&		4			\\
ECUs				&		13			\\
Memory				&		15GiB		\\
Instance Storage	&		2*40GB(SSD)	\\
Network Performance	&		High		\\
\end{tabular}
\caption{evaluation environment}
\label{evaluation:amazon_environment}
\end{table}

\begin{table}[th]
\centering
\begin{tabular}{|c|p{150pt}|}
\hline
CPU					&		Intel\textregistered Core\texttrademark i7-3770K CPU @ 3.50GHz\\\hline
Memory				&		16GB\\\hline
Storage				&		Crucial m4 CT256M4SSD3 (256GB, mSATA)(Peak read: 500 MB/s, Peak write:260MB/s)*8\\\hline
RAID Card 			&		Adaptec ASR-7805Q Single\\\hline
RAID				&		Raid 0\\
\hline
\end{tabular}
\caption{stroage environment}
\label{evaluation:stroage_environment}
\end{table}

here vCPUs means the number of virtual CPU in instance, and one ECU provides the equivalent CPU capacity of a 1.0-1.2GHz 2007 Opteron or 2007 Xeon processor.

For the storage system, we used a machine inside our lab, the details is shows in
Table.~\ref{evaluation:stroage_environment}.

All the compute nodes, I/O nodes and master node connect with interconnection network inside Amazon
EC2, and mount storage system by sshfs\cite{sshfs} via Internet.

\subsection{One User Performance}

\begin{figure}
\centering
\includegraphics[width=8.5cm]{img/one_client.pdf}
\caption{One User Performance}
\label{evaluation:one user performance}
\end{figure}

First we measure the performance of our prototype for only one user, and shows how our system
will affect one user performance.
In this experiment, the number of client is fixed to one, and measure
the sequential read and write performance for different number of I/O nodes.
All I/O data are distributed among all I/O nodes.
Figure~.\ref{evaluation:one user performance} shows one user performance.
As we know, one thread I/O is difficult to achieve the full throughput on Internet, we can see from
Figure.~\ref{evaluation:one user performance} that without I/O node, applications can only achieve a
throughput under 20MB/s in reading and under 80MB/s in writing.

However by using our system the read throughput increasing as the number of I/O nodes, and can
achieve about 8 times faster than the throughput without I/O nodes.
For writing throughput, the interconnection throughput is different from Internet throughput, it can
achieve full throughput even with one thread, but the throughput is limited by the client node
throughput which is also 1Gbps(135MB/s).

\subsection{Multiple Users Performance}

\begin{figure}
\centering
\includegraphics[width=8cm]{img/multiple_client.pdf}
\caption{Multiple User Performance}
\label{evaluation:multiple user performance}
\end{figure}

In the second experiment, we measured the throughput of multiple clients.
To simplified the experiment, we assume that one client connect to only one I/O nodes, and one I/O
nodes buffer all the I/O data for a client.
Figure.~\ref{evaluation:multiple user performance} shows the overall throughput of the whole system.
Since all the data are stored on the shared storage, and need to be transferred via Internet, the
read throughput doesn't change by using I/O nodes.
However when we look at the write throughput, it shows a strong scalability, and achieved 7 times
improvement with only 4 I/O nodes.

\subsection{Simulation for Applications}
Show our architecture can burst I/O performance for applications
x

%section VII
%to do 
\section{related work}
\label{sec:related work}

Burst buffer architecture are used in several previous research, as a new design in storage system
to fill the gap between compute node throughput and parallel file system throughput.
Ning Liu el at. \cite{on_the_role_of_burst_buffers} proposed burst buffers in high
performance(HPC) system to absorb bursty I/O request from applications, they analyzed the I/O
pattern of several applications, and the workload of the whole system, and propose a SSD based burst
buffer system to absorb such bursty I/O request instead of traditional approach which increase the
throughput of the file system and easily cause the under utilization.
Sato el at.\cite{checkpointing} proposed a burst buffer based checkpoint strategy to enable timely
checkpoint as well as alleviate the affect to applications.

The emergence of Cloud Computing technology bring a new solution to large scale computation for it's
economies of scale, low capital costs, high accessibility and flexibility.
There are already many research used public cloud for a large scale computation.
Shao el at.\cite{Geoprocessing_on_the_Amazon_cloud_computing_platform} showed a way to use Amazon
public cloud for geoprocessing.
Garcia, T el at.\cite{Analysis_of_Big_Data_Technologies_and_Method} run hadoop and open source
parsers on Amazon cloud for querying large web public RDF datasets.
Peter Wittek el at.\cite{XML_Processing_in_the_Cloud} combined an implementation-independent
workflowdesigner with cloud computing to support small institution in ad-hoc peak computing needs.
Richard Anthony el at.\cite{Cloud_computing_applications_for_large-scale_satellite_ground_systems}
discuss the applicability of cloud computing to large-scale satellite ground 
systems.
However these work only focus on how to run these large scale application on cloud, they didn't
consider the way to increase I/O throughput to shared storage.

I/O throughput problem is a critical part in cloud computing, and many work are focus on this
topic.
Abhishek Gupta el at.\cite{Towards_Efficient_Mapping} try to improve the execution of HPC
application in cloud by making a cloud-aware scheduler, and change the cloud environment
dynamically to improve the overall performance.
Matthias Hovestadt el at.\cite{Evaluating_Adaptive_Compression} consider another way to improve the
I/O performance.
They propose a new adaptive compression scheme for virtualized environments, and improve the I/O
performance by compress the I/O data.
Hongtao Du el at.\cite{DHFS:_A_High-Throughput_Heterogeneous_File_System_Based_on_Mainframe_for_Cloud_Storage}
focused on the performance of metadata server in cloud environment and propose a high throughput
system for Cloud Storage, by building metadata server on the high performance mainframe.
They also focused on the security part of the shared stroage in cloud, and propose a
high-throughput parallel file for secure Cloud Storage base on a new concurrent write
mechanism\cite{PsFS:_A_high-throughput_parallel_file_system_for_secure_Cloud_Storage_system}.


%section VIII
%to do
section VII: conclusion


\section*{Acknowledgement}
This work was performed under the auspices of the U.S. Department of Energy by
Lawrence Livermore National Laboratory under Contract DE-AC52-07NA27344.
(LLNL-CONF-663584-DRAFT).
\xtqmod{ This research was supported by JST, CREST (Research Area: Advanced Core Technologies for
Big Data Integration).}
\kento{Please add EBD Crest acknowledgment}
This research made use of Montage, funded by the National Aeronautics and Space Administration's
Earth Science Technology Office, Computation Technologies Project, under
Cooperative Agreement Number NCC5-626 between NASA and the California Institute
of Technology. Montage is maintained by the NASA/IPAC Infrared Science Archive.
\bibliography{src/ref/reference.bib}

\end{document}
