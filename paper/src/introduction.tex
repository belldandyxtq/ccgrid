\section{Introduction}
\label{sec:introduction}
\xtq{writing}
A growing focus on cloud computing for its high scalability as well as high computational resources and fast setup, 
Such cloud environment is suitable for large scale computation, for example, Amazon provided a high performance computing instance allows user to run complex science, engineering, and business applications on these instances with a high bandwidth low latency network, and high compute capablilies.
However, such applications often require high I/O throughput and data sharing between different nodes, different workflow and even different jobs.
In such cases, applications require not only high bandwidth network as well as a high throughput shared storage in order to share data between different workflow or difference jobs.
Applications like\xtq{applications like montage povray} generate \xtq{file size } intermediate file to shared between different nodes in different workflow.
Such applications are now running majorly on high performance computing system like supercomputers, with a GB/s level shared storage as well as high interconnection network.

However if we attempt to run such applications on public cloud, currently we can not get acceptable performance, since the lack of a shared storage system with a required throughput.
For example, in Amazon public cloud system, there are a famous shared storage called \emph{Amazon Simple Storage Service}(Amazon S3), but compared with the throughput of shared storage system in high performance computing system,
Amazon S3 can only achieve a extremely low throughput, \xtq{figure}as well as a high latency, since storage machine in Amazon S3 are geographically distributed and connected via Internet.
Also it is difficult to build each compute center a reasonable size data center, since it can not achieve a high utilization of storage system.

In order to solve such problem, we propose a cloud based burst buffer system. our proposal system take advantage of high throughput, low latency interconnection network inside cloud environment,
make some of compute nodes as a I/O burst buffer nodes to buffer I/O data and bursting I/O sensitive applications in cloud environment.

Our contributions can be summarized as following:
\begin{itemize}
	\item A cloud based I/O burst buffer system for increasing data transfer throughput in cloud environment;
	\item An evaluation of prototype of I/O burst buffer system in Amazon EC2;
    \item \xtq{queue model?}
\end{itemize}
%2
The rest of this paper is organized as follows. 
In Section \ref{sec:motivation}, we clarify the motivation and the background.
We introduce an overview of the I/O bursting buffer architecture in Section \ref{sec:architecture}, 
and describe the detail about the first prototype implementation of our proposal architecture in order to evaluate its performance in a real environment \ref{sec:implementation}. 
In Section \ref{sec:evaluation}, we present our experimental results based of our performance models. %based on data obtained from several benchmark on TSUBAME V queue and Amazon EC2 
Finally, we show some related work in Section \ref{sec:related work}, and conclusion in Section \ref{sec:conclusion}.
