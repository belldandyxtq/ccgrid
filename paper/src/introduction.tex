\section{Introduction}
\label{sec:introduction}
% Kento: Growing focus ``does not'' allows users to run ...
% A growing focus on cloud computing for its high scalability as well as high computational
% resources, for example, Amazon provided a high performance computing instance allows user to run
% complex science, engineering, and business applications on these instances with a high bandwidth low latency
% network, and high compute capabilities.
% Cloud provide a fast setup, usually cloud instances can set up in a few
% minutes, this make the on-demand usage available. 
% Such cloud environment is suitable for large scale
% computation, by using cloud users don't need a large scale supercomputer and
% only need to pay what they run, instead of dealing with expensive hardware, software and machine maintenance.
% These characteristics make cloud very attractive to large-scale applications.
Cloud have been gathering attention of application developers because
of its elasticity. With the elasticity, the users enjoy virtually unlimited
computational resources on the fly, and pay the cost depending on the usage.
In addition, recent clouds also provides computational resources for high
performance computing
\cite{AMAZON_AWS,Azure,IBM_public_cloud,oracle_cloud,hp_cloud}.
The computational power and the scalability enable the users to run high
performance scientific applications faster than ever.
For example, Amazon EC2 provides HPC instances with high bandwidth networks,
SSD-embedded I/O subsystems and GPU accelerators\cite{AWS_HPC}.
These characteristics make cloud computing more attractive to large scale
scientific applications.
\par
% Large-scale applications often require high I/O
% throughput and data sharing between different nodes, different work flow and even different jobs.
% In such cases, applications require not only high bandwidth network as well as a high throughput
% shared storage in order to share data between different nodes, different work flow or difference
% jobs.
% Applications like Montage\cite{montage} Pov Ray\cite{povray} generate huge size of intermediate file
% need to be shared between different nodes in different work flow.
% Such applications are now running majorly on high performance computing system like supercomputers,
% with a GB/s level shared storage as well as high interconnection network.
% Furthermore, large scale applications require for a large amount of compute node as well as a
% running time from several hours to several days, machine failure is Inevitable and checkpoint will be required to enable
% fast recovery from machine failure\cite{checkpointing}, in order to make a timely checkpoint, a
% fast global shared storage is indispensable.
% 
% However, in current cloud environment, the global shared storage can not meet the demand.
% For example, in Amazon public cloud system, there are a famous shared storage called \emph{Amazon
% Simple Storage Service} (Amazon S3)\cite{AMAZON_AWS}, however compared to the throughput of shared
% storage system in high performance computing system, Amazon S3 can only achieve a extremely low throughput,
% as well as has a high latency, since storage machine in Amazon S3 are geographically distributed and connected via Internet.
% Also it is difficult to build each compute center a reasonable size data center, since it cannot
% achieve a high utilization of storage system.

However, if we execute \emph{data intensive workloads}, such as big data
analysis and checkpointing for visualization or fault tolerance, on a cloud environment, the
users suffer from the prolonged execution time and the cost because of the low
throughput to shared storage and the data intensive workloads.
For example, cloud-based shared storage, Amazon Simple Storage Service (Amazon
S3), provides only a few hundreds MB/s of throughput because instances
in Amazon EC2 is connected to Amazon S3 via Internet.
As demands for higher I/O throughput grow, 
current cloud environments cannot provide enough I/O performance for 
\emph{data intensive workloads}.
Thus, improvement of I/O performance in clouds is critical to accelerate such
\emph{data intensive workloads}.
\par
% In order to solve such problem, we propose a cloud based burst buffer system. Burst buffer system
% has been proposed as a new tier in the storage hierarchy to hide the low bandwidth and high latency
% of the shared file system by allocate several local nodes as a burst buffer nodes to absorb the I/O
% request from applications, our proposal system using such burst buffer system and take advantage of
% high throughput, low latency interconnection network inside cloud environment, make some of compute
% nodes as a I/O burst buffer nodes to buffer I/O data and bursting I/O sensitive applications in cloud environment.
% To validate effectiveness of burst buffers in cloud, we implemented a prototype of our
% proposal system and evaluated on real public cloud environment,
% % as well as simulated with several applications.\xtq{we e}
% We achieved 3 times speed up on reading as well as 4 times speed up on writing with 8 I/O nodes, and
% achieved a up to 4 times speed up on execution of data intensive applications from simulation.
In order to solve the problem, we propose a cloud-based burst buffer system.
Burst buffers have been proposed as a new tier of storage in the storage
hierarchy in order to provide higher I/O throughput for shared storage by
absorbing bursty I/O requests from applications running on supercomputers
\cite{on_the_role_of_burst_buffers}.
Because most of HPC applications have data access
locality, if we cache the data, which is accessed in near future, on the burst
buffers, we can accelerate the access performance to the data
\cite{montage,povray}.
In addition, if we apply burst buffers to clouds, we can build on-demand burst
buffers by taking advantage of the elasticity in clouds. 
%\kento{local storage use ? in-memory}
To explore the effectiveness of burst buffers in clouds, 
we implement a prototype of a cloud-based burst buffer system, and models the
system to estimate execution time of data intensive applications.
Our experimental results using Amazon EC2 shows that we achieve 3 times of
improvement in read throughput, and 4 times in write throughput with 8 nodes of
burst buffers. Our simulation based on the model shows that we can
improve execution time of real data intensive applications by 1.5 times in execution time.
\par


Our contributions can be summarized as following:
\begin{itemize}
	\item A cloud-based burst buffer system accelerating data intensive
	workloads in cloud environments;
	\item Models estimating I/O throughput in a real cloud environment
	\item Improvement of I/O performance by using our cloud-based burst buffer
	system in Amazon EC2;
    \item Quantitative evaluations targeting real environments where
    multiple real applications run, and access to the cloud-based burst
    buffers;
\end{itemize}
%2
\par
The rest of this paper is organized as follows. 
In Section \ref{sec:background}, we describe the background and the motivation.
Then, we introduce an overview of the cloud-based burst buffer system in Section \ref{sec:architecture}, 
and describe the details about our prototype implementation in Section \ref{sec:implementation}.
Next, in Section \ref{sec:evaluation}, we present our experimental results
of the cloud-based burst buffer system.
Finally, we describe related work in Section \ref{sec:related work}, and
conclusion in Section \ref{sec:conclusion}.
