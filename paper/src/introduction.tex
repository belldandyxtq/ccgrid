\section{Introduction}
\label{sec:introduction}
A growing focus on cloud computing for its high scalability as well as high computational
resources, for example, Amazon provided a high performance computing instance allows user to run
complex science, engineering, and business applications on these instances with a high bandwidth low latency
network, and high compute capabilities.
Cloud provide a fast setup, usually cloud instances can set up in a few minutes, this
make the on-demand usage available.
Such cloud environment is suitable for large scale
computation, by using cloud users don't need a large scale supercomputer and only need to pay
what they run, instead of dealing with expensive hardware ,software and machine maintenance.
These characteristics makes cloud very attractive to large scale applications.

Large scale applications often require high I/O
throughput and data sharing between different nodes, different work flow and even different jobs.
In such cases, applications require not only high bandwidth network as well as a high throughput
shared storage in order to share data between different work flow or difference jobs.
Applications like Montage\cite{montage} Pov Ray\cite{povray} generate huge size of intermediate file
need to be shared between different nodes in different work flow.
Such applications are now running majorly on high performance computing system like supercomputers,
with a GB/s level shared storage as well as high interconnection network.
Furthermore, large scale applications require for a large amount of compute node as well as a
running time from several hours to several days, machine failure is Inevitable and checkpoint will be required to enable
fast recovery from machine failure\cite{checkpointing}, in order to make a timely checkpoint, a
fast global shared storage is indispensable.

However, In current cloud environment, the global shared storage can not meet the demand.
For example, in Amazon public cloud system, there are a famous shared storage called \emph{Amazon
Simple Storage Service}(Amazon S3)\cite{AMAZON_AWS}, however compared to the throughput of shared
storage system in high performance computing system, Amazon S3 can only achieve a extremely low throughput,
as well as has a high latency, since storage machine in Amazon S3 are geographically distributed and connected via Internet.
Also it is difficult to build each compute center a reasonable size data center, since it can not
achieve a high utilization of storage system.

In order to solve such problem, we propose a cloud based burst buffer system. Burst buffer system
has been proposed as a new tier in the storage hierarchy to hide the low bandwidth and high latency
of the shared file system by allocate several local nodes as a burst buffer nodes to absorb the I/O
request from applications, our proposal system using such burst buffer system and take advantage of
high throughput, low latency interconnection network inside cloud environment, make some of compute
nodes as a I/O burst buffer nodes to buffer I/O data and bursting I/O sensitive applications in cloud environment.
Our proposal system not only target on the work flow applications, all the applications have
I/O data locality or even output a huge size can get speed up on our proposal system.
We implemented a prototype of our proposal system and evaluated on real public cloud environment, as
well as simulated with several applications.
We achieved 8 times speed up on reading as well as 7 times speed up on writing with 8 I/O nodes, and
achieved a up to 4 times speed up on execution of data intensive applications from simulation.

Our contributions can be summarized as following:
\begin{itemize}
	\item A cloud based I/O burst buffer system for increasing data transfer throughput in cloud environment;
	\item A quantitative evaluation of prototype of I/O burst buffer system in Amazon EC2 to show the
	Effectiveness of our proposal system;
    \item A simulation with several applications in Amazon EC2 to show how applications can achieve
    performance burst by using our proposal system;
\end{itemize}
%2
The rest of this paper is organized as follows. 
In Section \ref{sec:background}, we clarify the motivation and the background.
We introduce an overview of the I/O bursting buffer architecture in Section \ref{sec:architecture}, 
and describe the detail about the first prototype implementation of our proposal architecture in
order to evaluate its performance in a real environment \ref{sec:implementation}.
In Section \ref{sec:evaluation}, we present our experimental results based of our performance models.
Finally, we show some related work in Section \ref{sec:related work}, and conclusion in Section \ref{sec:conclusion}.
