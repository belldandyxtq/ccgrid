\section{Modeling of The Cloud-based Burst Buffers}
\label{sec:modeling}
\subsection{Modeling of I/O throughput}
In practice, when we use the cloud-based burst buffers in real environments,
these burst buffers are shared, and accessed by multiple compute nodes.
Therefore, the I/O throughput can be degraded by other compute nodes.
To estimate the I/O throughput and simulate the overall execution
time of applications, we model the I/O throughput of the cloud-based buffer
buffers.

In our model, we make two assumptions: 
\begin{itemize}
  \item The master node evenly distributes data of applications across burst
  buffers nodes evenly so that I/O workloads are balanced across
  burst buffer nodes;
%  \item All I/O data are evenly distributed over all I/O nodes.
%  \item Applications access all data evenly.
  \item If multiple compute nodes access to a single burst buffer node, the
  bandwidth is divided by the number of compute nodes accessing to the node.
%  \item Multiple compute nodes can share the throughput of I/O nodes equally.
\end{itemize}
If a \emph{target} compute node access to a single burst buffer node(A), and the  
a single burst buffer is being by $i$ number of compute nodes, the actual
throughput of the target compute node, $T_{a}(i)$, becomes
$T_{a}(i)=\frac{T_{b}}{i+1}$ where $T_{b}$ is a peak I/O bandwidth of a single
burst buffer node. Thus, overall average I/O throughput, $T$, can be computed as:
\begin{equation}
\text{T}=\sum_{i=0}^{n-1} P_{i} \times T_{a}(i)
%\text{T}=\frac{T}{e} 
%e=1+\frac{n-1}{m}
\end{equation}
The $n$ denote total number of compute nodes, and $P_{i}$ denotes probability
where $i$ number of compute nodes accessing the burst buffer node(A). 
Hence, the $P_{i}$ can be computed as:


\begin{equation}
\text{P}_i=\frac{\dbinom{n-1}{i}(m-1)^{n-i-1}}{m^{n-1}}
\end{equation}

where $m$ denotes total number of burst buffer nodes. 
By using the model, we can estimate the throughput of applications in
environments where multiple compute nodes are running and randomly accessing the
burst buffer nodes.

\subsection{Modeling of Cost}
In addition to I/O throughput of the cloud-based burst buffers, another
important aspect is the cost.
While I/O throughput can be accelerated by burst buffers, the cost can be
increased because of additional nodes for burst buffers.
To estimate the cost, we also model the cost of the cloud-based burst buffers.

When we run an application using $\acute{n}$ number of compute nodes in a
environment where $b$ number of burst buffers are used, the cost is charged for
the compute node, and burst buffer nodes.
Thus, the total cost, $C_{total}$ can be computed as:   If a \emph{target}
compute node
\begin{equation}
C_{total} = \{ \acute{n} \times \text{C}_{cn}+ (b \times \text{C}_{bn} \times
 + 1 \times C_{mn}) \times \frac{\acute{n}}{n} \}\times etime
\end{equation}

where $C_{cn}$, $C_{bn}$ and $C_{mn}$ denote costs per second (${\it \$} /sec$)
of a compute node, a burst buffer node, and a metadata server node
respectively, $etime$ denotes the execution time of the application. If we
use the same instances for there nodes, the cost becomes 
$C_{cn}=C_{bn}=C_{mn}$.
For the fairness of the cost, the cost for burst buffers is charged depending on the number of compute nodes used by the application in this model. By using this
model, we estimate the overall cost of an application run. 


