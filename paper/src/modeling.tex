\subsection{modeling}
\label{sec:modeling}

In order to simulate the execution of the three applications shown in Table.~\ref{background:work
flow applications}
we introduce a model for simulate the workload of the whole system, and get the throughput of one
I/O node.
\subsection{Cost}

execute without burst buffers cost:
\begin{equation}
Cost_{S3} = \text{cost}_{compute node}+\text{cost}_{S3}
\end{equation}

execute with only one node Cost:
\begin{equation}
Cost_{BB} = \text{cost}_{compute node}+\text{cost}_{burst buffers}+\text{cost}_{S3}
\end{equation}

compute node cost can be calculate as:

\begin{equation}
\text{compute node}_{cost} = \text{compute node unit cost} \times \text{\# of compute
node} \ time \text{execution time}
\end{equation}

burst buffers cost can be calculate as, (here \# of burst buffer +1(master))
\begin{equation}
\text{cost}_{burst buffers} = \text{burst buffers} \times \text{(\# of burst buffers+1)} \times
\text{execution time}
\end{equation}

S3 cost can be calculate as:
\xtq{according to http://aws.amazon.com/ec2/pricing/, I/O to and from S3 IN THE SAME available zone
with compute nodes cost 0}
\begin{equation}
\text{S3}_{cost} = \text{I/O size}\times \text{unit cost}
\end{equation}


\subsection{Throughput of one I/O node}
Burst buffers throughput($BBT$) of one compute node.
\begin{equation}
BBT = \frac{T}{(1 + E(X))}
\end{equation}

here X: \# of other compute nodes accessing the same I/O node.
E(x) can be calculate as:

\begin{equation}
E(X) = \sum E(X_i)
\end{equation}

\begin{equation}
E(X_i) = 1 + (\frac{1}{m}) \times I0_ration(i)
\end{equation}



\subsection{different workload pattern}
