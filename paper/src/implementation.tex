\section{Implementation}
\label{sec:implementation}

\begin{figure}
\centering
\includegraphics[width=8cm]{img/prototype_overview}
\caption{overview of prototype}
\label{implemetation:overview of prototype}
\end{figure}

In order to evaluate our architecture in a real environment and compare with current architecture,
we implemented a prototype.
In our architecture there are several global data such as total number of burst buffer nodes,
buffered file information. need to be shared between all nodes.
And since in our proposal architecture, numbers of burst buffer nodes can be dynamically determined,
a global scheduler is needed to decide when and how nodes are added and removed, these decision
should be made based on global work load and local work load.
Also a scheduler is needed to handle the operation like re-balance of buffered data when nodes
addition, and data write back when nodes deletion.

For above reasons in this implementation, we adopted master-worker model, a master is designed to
store global information and interactive with client daemon, as well as serve as a scheduler.
The architecture of our prototype is described as Figure \ref{implemetation:overview of prototype}

In our architecture, there are three types of nodes:
\begin{itemize}
	\item A master node manages all IOnodes information, maintains a set of buffered file meta data,
	handles operation like IOnode addition and deletion etc., and interact with clients.
	\item There are several I/O nodes response for actually storing data.
	\item A daemon runs on each client machine serving for communication with
	master and IOnodes, in order to hide architecture detail from user.
\end{itemize}

The shared storage system are mounted on both master and I/O nodes, both of master and I/O nodes
have the same assess permission to the shared file system.

\subsection{File Chunk}
In our implementation, files are usually stored in more than one I/O node depends on the files'
size and numbers of I/O nodes available.
Files are divided into mutable-size \emph{chunks}, each I/O node store one chunk.
Such one I/O node one chunk design simplified our data layout design, and operations in I/O nodes addition and deletion.
For each chunk we adopted a \emph{dirty flag}, in order to decide whether this chunk has been modified.
In write back phase, only the modified chunk needs to be written back to storage.
This dirty flag can reduce the write back data size and thus reduce the Internet congestion.

Applications usually generate various sizes of files, since in our architecture, memory is precious
resources, more memory means we can buffer more files, mutable size chunk design avoid the
disadvantage of memory waste in fixed-size chunk design, and enable us just to allocate requested size of memory.
However such one I/O node one chunk design and mutable size has some disadvantages.
First master must maintain the chunk size of each file, this put a burden on master.
Second, for a large file the chunk size will become very large, cause a coarse-grained write back
control, and reduce the performance gained by dirty flag.
Finally, since the chunk size is decided by file size, and one I/O node store only one chunk, it is
difficult to change buffered file size without adding to or removing some I/O nodes from file.

\subsection{Master Node}
In our master-IOnode model, master node is the supervisor of the entire system,
master node is in charge of interactive with client, make file chunk placement, and handle file operations including file open, read and write.
Master node also manage the I/O nodes and schedule events like node registration at node setup phase, addition and deletion.
In order to master node majorly maintain following two data structures:
\begin{itemize}
  \item I/O nodes information: including node ip address, total memory and available memory.
  \item file meta data: including file path, file id, total file size, access control, chunk size,
  dirty flag and a map from chunks to I/O nodes.
\end{itemize}

Since only master has the global knowledge of file chunk placement, client must connect to master
to get these information.
However, in such master-worker model, master is easy to become a bottleneck of the whole system,
thus we have to minimize the master involvement in file operation, client asks master about the
file meta data, including the chunk placement information, then client caches these information and
connect to I/O nodes directly for sending or receiving data.

\subsection{Client daemon}

\begin{figure}[tb]
	\centering
	\includegraphics[width=8cm]{img/client_daemon}
	\caption{Client Daemon}
	\label{implementaion:client_daemon}
\end{figure}

In each compute node, there is a client daemon, which is a file system client used to buffer I/O
data, communicate with master nodes and I/O buffer nodes, including send I/O request and send or
receive I/O data.
Figure.~\ref{implementaion:client_daemon} is a illustrate of client daemon inside client nodes and
buffer queue in I/O burst nodes.
Users don't know and should not know the information like IP address of master nodes,  client daemon
is used to hide these information from users.
By using client daemon, application don't need to change their code, just recompile with a library.
As figure.~\ref{implementaion:client_daemon} shows, client daemon send open, close request to master node, and cache the file information including file chunk size, a map to I/O nodes.
When applications send read and write request, daemon send the request to corresponding I/O nodes
according to the cached information from master node, and transfer data with I/O nodes, finally
send to applications.

%When a user application issue a write request, I/O data will be buffered in that node by client
%daemon, when user close the file, call flush function or I/O data size exceed I/O server buffer
%size, client daemon will try to transfer I/O data to buffer queue in I/O burst buffer, if buffer
%queue in not full, I/O data will be sent to I/O burst buffer via interconnection network and can be
% seen among computing nodes in the same system after that.
%However if buffer queue is full, client daemon will block user application and wait until buffer
%queue is ready to receive new I/O data, causing a low I/O throughput.

%Similarly when user issue a read request, there are two conditions: required file is buffered in
% buffer queue in I/O burst buffer, or file is stored only in storage in another system.
%In the first cases, file can be transferred to computing nodes from buffer queue directly, and can
% achieve a high throughput.
%If data is not in buffer queue, then a read operation described below will be executed, since data
% must be transferred from storage in another system, in this case, throughput will depend on Internet condition and it is hard to achieve a high throughput.

\subsection{File Operations}

\begin{figure}
\centering
\includegraphics[width=8cm]{img/file_operation}
\caption{Details of File Operation}
\label{implementation:file operation}
\end{figure}

In this section we give more details about the file operation in our implementation, including
open, close, read and write.

When an application open a file, client daemon first send file path, access mode to master,
master return the file id and a map from file to I/O node,
for unbuffered file, master record the file information, and assign a file id, and then send the
file id to client daemon.

After file opened, applications can use file id to send read or write request, in order to reduce
master's burden, these subsequent request will go to corresponding I/O nodes instead by using the
map.
receive the request, for file opened for reading request, master get the file size from shared
storage, and assign several I/O nodes to this file according to file size.
As for file opened in only writing since the file need to be create, master get file size from
client daemon and assign I/O nodes after that master send I/O nodes ip address as well as the map
information from chunks to I/O nodes to client daemon, client cache these information.

\subsection{Limitation}
Since this is a prototype of our proposal system, there are several limitations.
First, in this prototype implementation, both master node and I/O nodes use one thread, it means
there is only one request can be responded at any time.
this design may cause master easily becoming a bottleneck when the number of client increase.
Also, since at any time one I/O node can transfer data with only one client, when the bandwidth of
I/O node larger than compute node, it will cause the I/O node under utilization.
However, there are some difficulties to change to a multiple threads version, for example, there
will have a dependency between two I/O request, also synchronization is needed among read and write
operations.

Second, one I/O node one data chunk design simplified master's design, but it also brings several
limitations
\begin{itemize}
  \item when application write data, 
\end{itemize}

Third, in the node addition and deletion, there is no load balance in our prototype, such limitation
will reduce the benefit of node addition.

