%to do
\section{evaluation}
\label{sec:evaluation}
\begin{figure}
\centering
%\includegraphics{}
\caption{Data Locality}
\label{evaluation:data locality}
\end{figure}

\begin{figure}
\centering
%\includegraphics{}
\caption{One User Performance}
\label{evaluation:one user}
\end{figure}

\begin{figure}
\centering
\caption{multi user}
\label{evaluation:multi user}
\end{figure}

In this section we introduce the evaluation result of our prototype implementation in Amazon EC2 public cloud.
The environment is shown as Table.\ref{evaluation:environment}
\begin{table}[h]
\centering
\begin{tabular}{|c|c|}
Region				&		Tokyo		\\
Instance Type		&		m3.xlarge	\\
vCPUs				&		4			\\
ECUs				&		13			\\
Memory				&		15GiB		\\
Instance Storage	&		2*40GB(SSD)	\\
Network Performance	&		High		\\
\end{tabular}
\caption{evaluation environment}
\label{evaluation:environment}
\end{table}

here vCPUs means the number of virtual CPU in instance, and one ECU provides the equivalent CPU capacity of a 1.0-1.2GHz 2007 Opteron or 2007 Xeon processor.
In the following experiment, we 
and the applications used are shown as below:
\begin{table}
\centering
\begin{tabular}{|c|}
montage\\
pov ray\\
supernovae\xtq{not exact}
\end{tabular}
\end{table}

\subsection{Data Locality}
Since applications I/O pattern will affect the performance gains by using our propose architecture
greatly, We measure the I/O pattern for the applications mentioned above.
Here we assume that all I/O data can be buffered in a single I/O node, otherwise data locality will
be affected by architecture data layout and master I/O node assignment strategy.
We count the duplicated read at the same position and all write as data locality size, since all
these cases data can be read from or write to buffer.
Figure~.\ref{evaluation:data locality} shows the results of these three applications. we can see
that the I/O pattern of all the three applications shows a strong data locality. Montage and Pov ray
are over 90\% and supernovae achieved over 80\%.
The reason is that these applications are work flow application, the whole job consists of several
sub jobs, the output of previous job can be the input of subsequent job, also some job will access
the same input file.
So if we buffer the output data and input data, the subsequent job can access the data via LAN.

\subsection{One User Performance}
First we measure the performance of our implementation for only one user.
In this experiment, fix the number of client to one, and measure the performance for different
number of I/O nodes. Figure~.\ref{evaluation:one user}