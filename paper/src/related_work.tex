\section{related work}
\label{sec:related work}

Burst buffer architecture are used in several previous research, as a new design in storage system
to fill the gap between compute node throughput and parallel file system throughput.
Ning Liu el at. \cite{on_the_role_of_burst_buffers} proposed burst buffers in high
performance(HPC) system to absorb bursty I/O request from applications, they analyzed the I/O
pattern of several applications, and the workload of the whole system, and propose a SSD based burst
buffer system to absorb such bursty I/O request instead of traditional approach which increase the
throughput of the file system and easily cause the under utilization.
Kento Sato el at.\cite{checkpointing} proposed a burst buffer based checkpoint strategy to enable
timely checkpoint as well as alleviate the affect to applications.

The emergence of Cloud Computing technology bring a new solution to large scale computation for it's
economies of scale, low capital costs, high accessibility and flexibility.
There are already many research used public cloud for a large scale computation.
Shao el at.\cite{Geoprocessing_on_the_Amazon_cloud_computing_platform} showed a way to use Amazon
public cloud for geoprocessing.
Garcia, T el at.\cite{Analysis_of_Big_Data_Technologies_and_Method} run hadoop and open source
parsers on Amazon cloud for querying large web public RDF datasets.
Peter Wittek el at.\cite{XML_Processing_in_the_Cloud} combined an implementation-independent
workflowdesigner with cloud computing to support small institution in ad-hoc peak computing needs.
Richard Anthony el at.\cite{Cloud_computing_applications_for_large-scale_satellite_ground_systems}
discuss the applicability of cloud computing to large-scale satellite ground 
systems.
However these work only focus on how to run these large scale application on cloud, they didn't
consider the way to increase I/O throughput to shared storage.

I/O throughput problem is a critical part in cloud computing, and many work are focus on this
topic.
Abhishek Gupta el at.\cite{Towards_Efficient_Mapping} try to improve the execution of HPC
application in cloud by making a cloud-aware scheduler, and change the cloud environment
dynamically to improve the overall performance.
Matthias Hovestadt el at.\cite{Evaluating_Adaptive_Compression} consider another way to improve the
I/O performance.
They propose a new adaptive compression scheme for virtualized environments, and improve the I/O
performance by compress the I/O data.
Hongtao Du el at.\cite{DHFS:_A_High-Throughput_Heterogeneous_File_System_Based_on_Mainframe_for_Cloud_Storage}
focused on the performance of metadata server in cloud environment and propose a high throughput
system for Cloud Storage, by building metadata server on the high performance mainframe.
They also focused on the security part of the shared storage in cloud, and propose a
high-throughput parallel file for secure Cloud Storage base on a new concurrent write
mechanism\cite{PsFS:_A_high-throughput_parallel_file_system_for_secure_Cloud_Storage_system}.
