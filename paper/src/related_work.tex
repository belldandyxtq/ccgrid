\section{related work}
\label{sec:related work}

Burst buffer is used in several previous researches as a new design in storage
system to fill the performance gap between node-local storage throghput and
parallel file system throughput.
Liu et al. \cite{on_the_role_of_burst_buffers} first proposed burst buffers for
high performance systems to absorb bursty I/O request from applications. 
They analyzed the I/O patterns of several applications, and the workload of the
whole system, and delivered the effectiveness.
Sato et al.\cite{checkpointing} explored effectiveness of burst buffers in
checkpointing, and proposed checkpoint strategy for burst buffers.

The emergence of a cloud computing technology brings a new solution to large
scale high performance computing with low initial costs, high accessibility and flexibility.
There are already several researches using public clouds for a large scale
computation.
Shao et al.\cite{Geoprocessing_on_the_Amazon_cloud_computing_platform} presented
that use of Amazone EC2 drives geoprocessing innovations.
Garcia et al.\cite{Analysis_of_Big_Data_Technologies_and_Method} deployed
\emph{Hadoop} on Amazon EC2 for querying large web public datasets.
Wittek et al.\cite{XML_Processing_in_the_Cloud} combined an
implementation-independent workflow designer with cloud computing to support small institution in ad-hoc
peak computing needs.
Anthony et
al.\cite{Cloud_computing_applications_for_large-scale_satellite_ground_systems}
discussed the applicability of cloud computing to large-scale satellite ground systems.
However, these work only focus on how to run these large scale application on
cloud, they didn't consider the way to increase I/O throughput to shared
cloud storage.

I/O throughput problem is a critical part in cloud computing, and many works
have been focusing on this problem.
Gupta et al.\cite{Towards_Efficient_Mapping} tried to improve the
execution of HPC applications in clouds by using a proposed cloud-aware
scheduler for dynamically changing cloud environment to improve the overall
I/O performance.
Hovestadt et al.\cite{Evaluating_Adaptive_Compression} considered another way to
improve the I/O performance.
They proposed a new adaptive compression scheme for virtualized environments,
and improved the I/O performance by compressing the I/O data.
Hongtao Du et
al.\cite{DHFS:_A_High-Throughput_Heterogeneous_File_System_Based_on_Mainframe_for_Cloud_Storage}
focused on the performance of metadata server in cloud environments. They
proposed a high throughput system for cloud Storage by building metadata server on the high performance mainframe.
%They also focused on the security part of the shared storage in clouds, and
%proposed a high-throughput parallel file system for secure cloud storage base
% on a new concurrent write mechanism\cite{PsFS:_A_high-throughput_parallel_file_system_for_secure_Cloud_Storage_system}.

Meanwhile, we innovate the state-of-the-art burst buffer technology for clouds
to accelerate I/O performance. To the best our knowledge, this work is the first explorations
of innovating burst buffer technologies in order to solve low I/O throughput
problems in clouds.
