%to do 
\section{related work}
\label{sec:related work}

\subsection{Burst Buffer}
Ning Liu\emph{el at.} \cite{on_the_role_of_burst_buffers} proposed a new tier of solid-state burst
buffers in high performance(HPC) system, 

\subsection{Burst Communication Throughput in Cloud Bursting}
Cloud computing is becoming a topic of much interest in recent years, not only famous Internet
companies like google, Amazon, IBM, Oracle, Microsoft provide public cloud, many companies start to
build or have built private cloud for internal computation.
%Several research works have been done on hybrid cloud and cloud bursting, Tekin Bicer \emph{el
%at.}\cite{time_and_cost} considered a software framework to enable data-intensive computing with
%cloud bursting, which use a combination of compute resources from local cluster and a public cloud
%to processing on a geographically distributed data set.
Their framework assume computation nodes allocated in both local cluster and public cloud, and data
set is geographically distributed, However in our study, data set is stored in local system and in
order to obtain a high communication throughput between computation nodes, we assume that nodes
used for the same job allocated in the same system.
Another cloud bursting application can be seen in Tian Guo\emph{el at.}\cite{Seagull}, they
introduced a system called Seagull, designed to facilitate cloud bursting by determining which
applications should be transitioned into the cloud and automating the movement process at the proper
time. Their work focused on determine which applications should be moved to public cloud, and our
work focus on the methodology of filling the I/O throughput in cloud bursting.