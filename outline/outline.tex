\documentclass{article}
\begin{document}
\title{Cloud-based I/O burst buffer}
\maketitle

\section{Introduction}
\begin{itemize}
	\item introduce cloud computation and show the problem in data sharing through shared storage.
	\item compare supercomputer application execution time with cloud execution time, show how bad it is for data sensitive application.
	\item purpose a cloud based burst buffer to burst IO throughput between compute node and shared storage in cloud computation.
\end{itemize}

\section{Background}
\subsection{a overview of cloud computation}
introduce common cloud architecture, consist of compute node and shared storage, and shared storage system is globally distributed and connected via Internet causing low throughput.
\subsection{a general view of burst buffer}
\subsection{introduction of some data sensitive applications}
	\begin{itemize}
		\item montage
		\item supernoveas
		\item povray etc.
	\end{itemize}
\subsection{introduction of AWS (experiment environment)}

\section{Architecture}
\subsection{a overview of cloud based IO burstbuffer}
\subsection{two IO patterns (just like SWoPP paper)}

\section{Implementation}
\subsection{a overview of implementation}
\begin{itemize}
	\item a master manages all IOnodes info and maintain a namespace and file metadata, handle operation like IOnode addtion and deletion
	\item several IOnodes response to actually store data.
	\item a client connect with master to get file meta info and connnect to IOnodes to transfer data
\end{itemize}

\subsection{master}
functionality:

	\begin{itemize}
		\item manage IOnodes including addtion, deletion.
		\item mangee data layout, including load balance, and data rebalance when IOnodes addtion and deletion.
		\item maintain namespace of buffered data.
		\item interact with client
	\end{itemize}

\subsection{IOnode}
functionality:

\begin{itemize}
	\item buffer client output data.
	\item read data from storage.
\end{itemize}

\subsection{file operation}
\begin{itemize}
  \item open
  \item read
  \item write
  \item close
\end{itemize}

\section{Evaluation}
\subsection{Evaluation of Implementation}
evaluation result published in SWoPP and first implementation evaluation result showed at meeting with amazon people.
\subsection{Data Locality Evaluation}
montage etc. data locality results
\subsection{Queuing model}
		
\section{Related Work}

\section{Conclusion}
\end{document}
